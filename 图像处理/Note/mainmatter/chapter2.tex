\chapter{Photoslip Algorithm: 基于深度学习}

本篇文章来自\url{https://zhuanlan.zhihu.com/p/660866515?utm_campaign=shareopn&utm_medium=social&utm_psn=1815096437689683968&utm_source=wechat_session}

\section{Flow Net}

采用端对端的学习方法来预测光流:给定一个由图像对和光流的\textsl{ground truth}组成的数据集,训练一个卷积神经网络直接从图像对中预测光流信息。

假设输入端为两幅序列图像$I_1,I_2$,输出端为预测的光流$F$
\begin{equation}
    F=CNN(\theta, I_1, I_2)
\end{equation}

$\theta$为待学习参数。

\subsection*{\textsl{网络结构}}

在\textsl{FlowNet}中,使用的网络为\textsl{Unet},分为\textsl{Decoder}和\textsl{Encoder}。

根据输入方式的不同,\textsl{FlowNet}提出了两种网络结构

\begin{enumerate}[itemindent=2em]
    \item \textsl{FLowNetSimple}
    
    直接将两张图像按通道重叠后输入,\textsl{channel}为$6$。

    \item \textsl{FlowNetCorr}
    
\end{enumerate}

\subsection*{\textsl{解码器}}

\subsection*{\textsl{总结}}

首个将CNN网络用于光流估计,精度和效果比不上传统基于能量最小化的方法。

\section{PWC Net}

\subsection*{\textsl{原理}}

采用特征金字塔结构作为特征提取网络,其包含三个部分
\begin{enumerate}[itemindent=2em]
    \item \textsl{pyramid};
    \item \textsl{wrap};
    \item \textsl{cost};
    \item \textsl{volume};
\end{enumerate}

\subsection*{\textsl{网络结构}}

\begin{enumerate}[itemindent=2em]
    \item \textsl{pyramid feature extractor};
    \item \textsl{wraping layer};
    \item \textsl{cost volume layer};
    \item \textsl{optical flow estimator};
    \item \textsl{context networks}
\end{enumerate}

\subsection*{\textsl{损失函数}}

训练使用$L_2$损失
\begin{equation}
    \mathcal{L}(\Theta)=\sum^{L}_{l=l_0}\alpha_l\sum_{\mathbf{x}}\left| \mathbf{w}^l_\Theta(\mathbf{x})-\mathbf{w}^l_{GT}(\mathbf{x}) \right|_2+\gamma|\Theta|_2
\end{equation}

其中,$\Theta$为不同金字塔层的特征提取参数和光流估计参数,$\mathbf{w}^l_\Theta(\mathbf{x})$表示网络预测的第$l$金字塔层的光流,$\mathbf{w}^l_{GT}(\mathbf{x})$表示标签值。

\subsection*{\textsl{总结}}

结合了\textsl{coarse-to-fine}与传统方法中的多种思想,效果优于传统方法,但是模型参数多,计算量大。

\section{IRR}

\subsection*{\textsl{原理}}

受传统方法能量最小化的启发,基于共享权重,对光流进行迭代化细化,利用\textbf{双向估计}和\textbf{遮挡推理}提高光流精度。

\subsection*{\textsl{网络结构}}

\begin{enumerate}[itemindent=2em]
    \item \textsl{FlowNetS}:在不增加参数或使训练过程复杂化的情况下,通过迭代地重复使用单个网络来不断优化先前的光流估计
    \begin{equation}
        \mathbf{f}^i_{fw}=D(E(\mathbf{I}_1),\omega(E(\mathbf{I}_2),\mathbf{f}^{i-1}_{fw}))+\mathbf{f}^{i-1}_{fw}
    \end{equation}

    \item \textsl{Based PWC-Net}:用一个共享解码器来替代多个解码器,该解码器迭代地改进所有金字塔级别的输出;
    \item \textsl{遮挡估计}:在\textsl{encoder}末端的第一帧增加了一个额外的\textsl{decoder}$\mathbf{o}^{i}_1$估计遮挡,与光流decoder平行,$\mathbf{o}^{i}_1$的输入与光流的\textsl{decoder}相同,输出通道的数量为$1$;
    \item \textsl{双向估计}:
    \item \textsl{对光流和遮挡的双边优化}:
    \item \textsl{occlusion上采样层}
\end{enumerate}

\subsection*{\textsl{损失函数}}

分别对光流和遮挡定义$L_2$损失和交叉熵损失

\subsection*{\textsl{总结}}

双向估计对于光流精度的提升不高,但利用正向和反向的一致性能够估计更加准确的遮挡;

\section{RAFT}

为了优化光流估计,RAFT提取像素特征,对所有像素建立多尺度correlation volume,在correlation volume上进行邻域采样得到lookups特征(增强特征相关性,也可以理解为感受野),之后直接使用以CNN-GRU为基础的迭代优化网络,在完整尺寸上对光流估计迭代优化。

\section{GMA}

遮挡问题是指像素点在第一帧中存在,但在第二帧(匹配帧)中不存在的问题。以往的算法使用cost volume去计算两帧之间的相关性,但这种相关性对于被遮挡区域来说是毫无意义的。当两帧像素之间不存在匹配信息时(第二帧像素在第一帧中不存在),单个物体的运动通常时均匀的,运动信息必须从其他像素开始传播。也就是说,未被遮挡的像素点的运动信息可以传播到被遮挡的像素点。但利用传统卷积进行局部传播的缺点在于其范围有限,因此提出global motion aggregation GMA模块(全局聚合模块),以Raft为基础,利用图像自相似解决遮挡问题,该模块基于transformer,可以更好的去学习两帧像素之间的依赖关系(哪个像素与它有关或它属于哪个物体)。

